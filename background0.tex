\section{Jiří's approach}

\subsection{Preliminaries}

\iffalse
* A graph <V,E>
* A set of agent A, defined by a starting position and an ending position in V.
* A shortest path is an acyclic path that start in the starting position of an agent and end in the ending position of the same agent.
- A solution is a plan, aka a set of tuple <A,T> where A is an action (referencing the agent and the movement) and T the time-point at which the action is executed.
- A solution is a valid solution if there is no conflicts and the execution of the actions at the related time-points lead to a correct last state.
  - A conflicts is either a edge conflicts or a vertex conflict.
  - An edge conflict is a pair of actions <<A_1,E1>,T>,<<A_2,E2>,T> such that E1=-E2.
  - A vertex conflict is complex.
- (note: waits are not represented as a tuple in the plan)
- A solution follow the shortest paths if all actions in the plan are referencing a movement of the same agent, and vice-versa.
- If a solution follow the shortest paths, it leads to the correct last state. As such, it just need no be free of conflicts to be valid.
\fi

\subsection{Agent's ordering}

\iffalse
Given a total ordering of the agent, given two agents $a_1$ and $a_2$ such that $a_1<a_2$, then for all tuple $\langle (a1,m1),t1 \rangle$ and all tuple <ac2,t2> referencing a_2: t1<t2.*

To assure vertice conflict, we have to observe the starting position and ending positions.
- Given two agent a1 and a2, if the starting position of a1 is in the shortest path of a2, then a1<a2.
- Given two agent a1 and a2, if the end position of a1 is in the shortest path of a2, then a1>a2.

Two different agents cannot do actions at the same time, so there is no edge conflict.

The final plan given by this approach is very unoptimized, because every parallel actions have been forbidden.
Also, of course, this is not complete.
\fi

\subsection{Partial ordering}

\iffalse
We can improve the approach by using a partial ordering instead of a total one.

To ensure that the possible parallels actions doesn't collide, we can add another constraint:
* If a vertice is in the shortest paths of two agents a1 and a2, a1>a2 or a1<a2.

This way, agents that doesn't share vertices doesn't need to wait for one-another.

Two differents agent that share an edge cannot do actions at the same time, so there is no edge conflict.
\fi

\subsection{Cycle solving}

\iffalse
Using the previous approach, we represent the ordering of the agents with a wait-for graph.

If a cycle is detected in this graph, we can look for a point to wait. One agent does the first part of his plan, then wait for all of the second agent, then finish his plan.

Otherwise, we look for a detour.

Otherwise we are screwed.
\fi

\iffalse
\subsection{My formalization of it}

Every shortest paths can be divided in sub-paths as follow:
* Every sub-path must be a path. This mean a sub path only contains succesives edges.
* All edges from the shortest path of an agent must be in one and only one sub-path of this agent.

From this, we can use the wait-for graph on the sub-paths.

We need to check starting and ending points of

\fi
