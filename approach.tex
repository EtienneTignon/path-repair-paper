\section{Approach}\label{sec:approach}

\subsection{Repair and pseudo conflict}
Given a pair of acyclic path $(p_1,p_2)$ we can define intersections called "pseudo-conflicts".

\begin{itemize}
  \item There is a "vertex pseudo-conflict" $(a_1,a_2,v)$ if $v \in a_1$ and $v \in a_2$.
%  \item There is a "edge pseudo-conflict" $(a_1,a_2,v,v')$ if $(v,v') \in a_1$ and $(v,v') \in a_2$.
  \item There is a "edge pseudo-conflict" $(a_1,a_2,v,v')$ if $(v,v') \in a_1$ and $(v',v) \in a_2$.
\end{itemize}

Given a plan\comment{Not any plan. Check the necessary properties.} $\pi$, you can define the associated path $p$. For simplicity, we can talk about the pseudo-conflict of plans to refer to the pseudo-conflicts of their paths.

If two plans have a conflict, their respective paths have the corresponding pseudo-conflict. The opposite is not true. For example, two plans taking the same action at different time does not have a conflict on this action, but the respective paths will have an edge pseudo-conflict.

Given a set of plans, we can define a "repaired set of plan" as follow:
\begin{itemize}
  \item The repaired set of plan does not have pseudo-conflict not present in the original set.\footnote{Before I forgot: if I forbid vertex pseudo-conflict, the other one may be avoided as well.}
  \item For each pseudo-conflict, we forbid the corresponding conflict.
\end{itemize}
Thanks to the pseudo-conflict, we know that no new conflict can arise. As such, we do not need to consider conflicts outside of theses.

\subsection{Constraint graph}
Given a MAPF instance $I=(V,E,A)$.
Given a set of shortest path\footnote{Acyclic path is enough.} $S = \{sp_a | a \in A \}$ that contain a shortest path for each agent.

We can represent a solution for $I$ using the set $S$ with a constraint graph $G_C=(V_C,E_C).$
\begin{itemize}
  \item $V_C = \{action(a,e) | sp_a \in S, e \in sp_a \}$
  \item $E_C = E_s \cup E_{C-ep} \cup E_{C-vp}$
\end{itemize}
The vertices of the graph represents the actions the agents have to execute.\footnote{Note that no "wait" action is present. We considers an agent may not do an action in all time-step. Not doing an action at T is equivalent with waiting at T.}
The edges represents the constraints between the time-points of two actions. $SE$ contains the edges that represent the succession of actions for each agent. $CE$ represent the edges

\begin{itemize}
  \item For every succeeding edges $(v1,v2),(v2,v3)$ in $sp_a$, we add in $E_s$ the edge $(action(a,(v1,v2)),action(a,(v2,v3)),-1)$ is in $SE$.
  \item For every vertex pseudo-conflict $(a_1,a_2,v)$ in $S$, we must add one of the following edge in $E_{C-vp}$:
  \begin{itemize}
    \item $(action(a_1,(v,v')),action(a_2,(v'',v)),0)$\comment{Add that the actions must be from thoses that allready exist}
    \item $(action(a_2,(v,v')),action(a_1,(v'',v)),0)$
  \end{itemize}
  \item For every edge pseudo-conflict $(a_1,a_2,v,v')$ in $S$, we must add one of the following edge in $E_{C-ep}$:
  \begin{itemize}
    \item $(action(a_1,(v,v')),action(a_2,(v',v)),-1)$
    \item $(action(a_2,(v',v)),action(a_1,(v,v')),-1)$
  \end{itemize}
\end{itemize}

From this given
