\section{Pseudo-conflict's approach}\label{sec:approach}

\subsection{Repair and pseudo conflict}
Given a pair of acyclic path $(p_1,p_2)$ we can define intersections called "pseudo-conflicts".

\begin{itemize}
  \item There is a "vertex pseudo-conflict" $(a_1,a_2,v)$ if $v \in a_1$ and $v \in a_2$.
%  \item There is a "edge pseudo-conflict" $(a_1,a_2,v,v')$ if $(v,v') \in a_1$ and $(v,v') \in a_2$.
  \item There is a "edge pseudo-conflict" $(a_1,a_2,v,v')$ if $(v,v') \in a_1$ and $(v',v) \in a_2$.
\end{itemize}

Given a stroll $\pi$, you can define the associated path $p$. For simplicity, we can talk about the pseudo-conflict of stroll to refer to the pseudo-conflicts of their paths.

From now on, the stroll will be called plan.

If two plans have a conflict, their respective paths have the corresponding pseudo-conflict. The opposite is not true. For example, two plans taking the same action at different time does not have a conflict on this action, but the respective paths will have an edge pseudo-conflict.

Given a set of plans, we can define a "repaired set of plan" as follow:
\begin{itemize}
  \item The repaired set of plan does not have pseudo-conflict not present in the original set.\footnote{Before I forgot: if I forbid vertex pseudo-conflict, the other one may be avoided as well.}
  \item For each pseudo-conflict, we forbid the corresponding conflict.
\end{itemize}
Thanks to the pseudo-conflict, we know that no new conflict can arise. As such, we do not need to consider conflicts outside of theses.

\subsection{Constraint satisfaction on actions}
Given a MAPF instance $I=(V,E,A)$.
Given a set of shortest path\footnote{Acyclic path is enough.} $S = \{sp_a | a \in A \}$ that contain a shortest path for each agent.

We can represent a solution for $I$ using the set $S$ with a CSP $\langle X,C \rangle.$
\begin{itemize}
  \item $X = \{(a,e) | sp_a \in S, e \in sp_a \}$ is the set of variables
  \item $C = E_s \cup E_{ec} \cup E_{vc}$ is the set of constraints
\end{itemize}
The domain of every variable is $ {\displaystyle \mathbb {N} }$.
The variables represents the actions the agents have to execute.
The constraints in $E_s$ represent the succession of actions for each agent. Those in $E_{ec}$ and $E_{vc}$ represent constraints between actions of a pseudo-conflict.

\begin{itemize}
  \item For every succeeding edges $(v1,v2),(v2,v3)$ in $sp_a$:
  \begin{itemize}
    \item $(a,(v2,v3)) - (a,(v1,v2)) \leq -1$ is in $E_s$.
  \end{itemize}
  \item For every vertex pseudo-conflict $(a_1,a_2,v)$ in $S$, we must add one of the following constraints in $E_{vc}$:\comment{Add that the actions must be from those that already exist. If none of the two can be added, there is no solution to the mapf instance using those shortest paths.}
  \begin{itemize}
    \item $(a_2,(v'',v)) - (a_1,(v,v')) \leq 0$
    \item $(a_1,(v'',v)) - (a_2,(v,v')) \leq 0$
  \end{itemize}
  \item For every edge pseudo-conflict $(a_1,a_2,v,v')$ in $S$, we must add one of the following constraints in $E_{ec}$:
  \begin{itemize}
    \item $(a_2,(v',v)) - (a_1,(v,v')) \leq -1$
    \item $(a_1,(v,v')) - (a_2,(v',v)) \leq -1$
  \end{itemize}
\end{itemize}

A solution of the mapf instance is obtained from a solution of the CSP:
\begin{itemize}
  \item For each variable $(a_i,(v_j,v_k)) \in X$, $X=t \rightarrow \pi_{a_i}[t]=(v_j,v_k)$
  \item The rest of $\pi_{a_i}$ is filled with waits \comment{TODO: formalisation}
\end{itemize}
\comment{TODO : getting the stroll}
