\section{Vertex ordering}\label{sec:vo}

Given a set of shortest path $sp_a$, we can define all visits of agents $a\in A$ on each vertex.

\begin{enumerate}
  \item $V = \{v_i : v = sp_i(k), k \in 0..|sp_i|\}$
  \item If $v = sp_i(k)$ and $v' = sp_i(k+1)$ for some $k \in 0..|sp_i|$, then $v_{i}<v'_{i}$.\label{succ}
  \item If $v \in sp_i$ and $v \in sp_j$, then $v_{i}<v_{j}$ or $v_{i}>v_{j}$.
  \\ This forbid vertex conflicts.\label{vercon}
  \item If $(v,v') \in sp_i$ and $(v,v') \in sp_j$ and $v_{pi_i} < v_{pi_j}$, then $v_{pi_i} < v_{pi_j}$. \\ This ensure that, if one agent follow another, he can't pass it.\comment{ET: redudant?}
  \item If $(v,v') \in sp_i$ and $(v',v) \in sp_j$ and $v_{i} < v_{j}$, then $v'_{i} < v'_{j}$ \\ This forbid edge conflicts.\label{edgcon}
  \item If $sp_i(0)=v$ and $v \in sp_j$, then $v_{i} < v_{j}$. \\ This ensure that an agent must leave his starting position for every other agent to pass on it.
  \item If $sp_i(|sp_i|-1)=v$ and $v \in sp_j$, then $v_{i} > v_{j}$. \\ This ensure that an agent must reach his goal after every other agent passed on it.
\end{enumerate}

This ordering can be represented as a waiting graph.

Note: in the "conflict-free-routing" repository, this is how the acyclicity checker works.

\subsection{Proof}

We can define a plan $\Pi = \{\pi_i : i\in A\}$ from the ordering.

\begin{itemize}
  \item $\forall k\in \mathbb{N}, \pi_i(k)\in\{v:v_j\in V, j=i\}.$
  \item $\forall v_i\in V, \exists k \in \mathbb{N}, \pi_i(k)=v_i$.
  \item If $v_{i}<v'_{j}$ then $\forall k,k'\in \mathbb{N} : (\pi_{i}(k)=v, \pi_{j}(k')=v') \rightarrow k<k'$.
\end{itemize}
\begin{proof}
$\Pi$ is a valid non-conflicting plans.

\begin{itemize}
  \item Using \ref{succ}, we can affirn that $\pi$ define a stroll. \comment{TODO: formalize}
  \item Using \ref{vercon}, we can affirm that $\forall k\in \mathbb{N}, \pi_i(k) \not= \pi_j(k)$
  \item For the edge constraints.
  \begin{itemize}
    \item Let's assume we have an edge constraint : $\exists c, \pi_i(c)=v, \pi_i(c+1)=v',\pi_j(c)=v', \pi_i(c+1)=v$
    \item $\pi_i(c)=v, \pi_i(c+1)=v' \rightarrow (v,v') \in sp_i$
    \item $\pi_j(c)=v', \pi_j(c+1)=v \rightarrow (v',v) \in sp_j$
    \item Using \ref{vercon}, $v_i < v_j$ and $v'_i > v'_j$
    \item But this contradict \ref{edgcon}.
  \end{itemize}
\end{itemize}

\end{proof}
